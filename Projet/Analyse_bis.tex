% Options for packages loaded elsewhere
\PassOptionsToPackage{unicode}{hyperref}
\PassOptionsToPackage{hyphens}{url}
%
\documentclass[
]{article}
\usepackage{amsmath,amssymb}
\usepackage{iftex}
\ifPDFTeX
  \usepackage[T1]{fontenc}
  \usepackage[utf8]{inputenc}
  \usepackage{textcomp} % provide euro and other symbols
\else % if luatex or xetex
  \usepackage{unicode-math} % this also loads fontspec
  \defaultfontfeatures{Scale=MatchLowercase}
  \defaultfontfeatures[\rmfamily]{Ligatures=TeX,Scale=1}
\fi
\usepackage{lmodern}
\ifPDFTeX\else
  % xetex/luatex font selection
\fi
% Use upquote if available, for straight quotes in verbatim environments
\IfFileExists{upquote.sty}{\usepackage{upquote}}{}
\IfFileExists{microtype.sty}{% use microtype if available
  \usepackage[]{microtype}
  \UseMicrotypeSet[protrusion]{basicmath} % disable protrusion for tt fonts
}{}
\makeatletter
\@ifundefined{KOMAClassName}{% if non-KOMA class
  \IfFileExists{parskip.sty}{%
    \usepackage{parskip}
  }{% else
    \setlength{\parindent}{0pt}
    \setlength{\parskip}{6pt plus 2pt minus 1pt}}
}{% if KOMA class
  \KOMAoptions{parskip=half}}
\makeatother
\usepackage{xcolor}
\usepackage[margin=1in]{geometry}
\usepackage{graphicx}
\makeatletter
\def\maxwidth{\ifdim\Gin@nat@width>\linewidth\linewidth\else\Gin@nat@width\fi}
\def\maxheight{\ifdim\Gin@nat@height>\textheight\textheight\else\Gin@nat@height\fi}
\makeatother
% Scale images if necessary, so that they will not overflow the page
% margins by default, and it is still possible to overwrite the defaults
% using explicit options in \includegraphics[width, height, ...]{}
\setkeys{Gin}{width=\maxwidth,height=\maxheight,keepaspectratio}
% Set default figure placement to htbp
\makeatletter
\def\fps@figure{htbp}
\makeatother
\setlength{\emergencystretch}{3em} % prevent overfull lines
\providecommand{\tightlist}{%
  \setlength{\itemsep}{0pt}\setlength{\parskip}{0pt}}
\setcounter{secnumdepth}{-\maxdimen} % remove section numbering
\ifLuaTeX
  \usepackage{selnolig}  % disable illegal ligatures
\fi
\IfFileExists{bookmark.sty}{\usepackage{bookmark}}{\usepackage{hyperref}}
\IfFileExists{xurl.sty}{\usepackage{xurl}}{} % add URL line breaks if available
\urlstyle{same}
\hypersetup{
  pdftitle={Linear Time Series Assignment},
  pdfauthor={Tony LAUZE; Tien-Thinh TRAN-THUONG},
  hidelinks,
  pdfcreator={LaTeX via pandoc}}

\title{Linear Time Series Assignment}
\usepackage{etoolbox}
\makeatletter
\providecommand{\subtitle}[1]{% add subtitle to \maketitle
  \apptocmd{\@title}{\par {\large #1 \par}}{}{}
}
\makeatother
\subtitle{ARIMA modelling of a time series}
\author{Tony LAUZE \and Tien-Thinh TRAN-THUONG}
\date{2024-04}

\begin{document}
\maketitle

\hypertarget{part-i-the-data}{%
\section{Part I : the data}\label{part-i-the-data}}

\hypertarget{what-does-the-chosen-series-represent}{%
\subsection{1 - What does the chosen series represent
?}\label{what-does-the-chosen-series-represent}}

For this project, we chose to work on the time series of the mineral and
other bottled waters production. The series is a French Industrial
Production Index (IPI) series, taken from the INSEE's time series
databank. More precisely, we will work on the series that is corrected
from seasonal variations and working days (CVS-CJO), covering the
1990/01-2024/01 period with a monthly frequency, resulting in 410
observations. The index is with base 100 in 2021. Both the raw and the
corrected series can be observed below (see \textbf{Figure 1}).

\begin{figure}
\centering
\includegraphics{Analyse_bis_files/figure-latex/unnamed-chunk-2-1.pdf}
\caption{Figure 1: Comparison of Raw and Corrected Series}
\end{figure}

On the raw series, a clear seasonality can be seen, as well as a slight
trend. On the corrected series, the seasonality seems to be gone (which
is all the point of the CVS-CJO), and an increasing trend can be
observed, which might be linear, from around 1995 to 2017. Apart from a
peak in the summer of 2003, which may be attributed to the notorious
heatwave that took place in France at that time, no outliers or
particular periods can be found in the series.

\hypertarget{transform-the-series-to-make-it-stationary-if-necessary.}{%
\subsection{2 - Transform the series to make it stationary if
necessary.}\label{transform-the-series-to-make-it-stationary-if-necessary.}}

Given no seasonality seems to arise, and no change in the variability is
noticeable through time, we only need to deal with the trend. A
possibility would be to estimate the trend that occurs from around 1995
to 2017 (by supposing it is linear) and then to remove it from the
series. However, this trend does not seem to continue after 2017, and
before 1995 the series is rather decreasing. Therefore, we will instead
try to differentiate the series, that is, to apply the operator
\(\Delta U_t = U_t - U_{t-1}\). The result can be seen in \textbf{Figure
2}.

\includegraphics{Analyse_bis_files/figure-latex/unnamed-chunk-4-1.pdf}
The one-time differentiated series seems to be centered around zero, no
clear change in variability can be seen: this series looks stationary.
To check if it is, we will proceed to some tests.

First, we can check whether a unit root is present in our series, using
the augmented Dickey-Fuller (ADF) test. Regarding the specification of
the test, the series being centered around zero and presenting no trend,
we use the regression with no intercept (constant) nor time trend. To
check the hypothesis of absence of trend, we can make a simple
regression of the series on time index (see the regression table in
\textbf{Table 1}): both the intercept and coefficient of the time index
are almost equal to zero and are definitely non significant. We can thus
be confident about the specification of the ADF test. Regarding the
number of lags, 7 are required so that the Ljung-Box test cannot reject
the null-hypothesis of absence of auto-correlation among the residuals
of the model until order 24. The results of the performed ADF test is
presented in \textbf{Table 2}, and it leads to reject the null
hypothesis of the presence of a unit root; which is in favor of
stationary.

\begin{table}
  \centering
  \caption{Regression of the series on time}
  \label{}
  \begin{tabular}{lc}
    \hline
    \hline
    & \textit{Dependent variable:} \\
    \cline{2-2}
    & diff_prod_water \\
    \hline
    index & 0.0004 (0.003) \\
    Constant & $-$0.017 (0.641) \\
    \hline
    Observations & 409 \\
    $R^{2}$ & 0.00004 \\
    \hline
    \hline
    \textit{Note:} & $^{*}p$
  \end{tabular}
\end{table}

``` Then, we can perform a KPSS test in which the null hypothesis is the
level-stationarity of the series. The p-value being greater than 0.1
(see \textbf{Table 3}), we cannot reject the null-hypothesis; which is
once again in favor of stationarity.

\hypertarget{graphically-represent-the-chosen-series-before-and-after-transforming-it.}{%
\subsection{3 - Graphically represent the chosen series before and after
transforming
it.}\label{graphically-represent-the-chosen-series-before-and-after-transforming-it.}}

\hypertarget{part-ii-arma-models}{%
\section{Part II : ARMA models}\label{part-ii-arma-models}}

\hypertarget{pick-and-justify-your-choice-an-armapq-model-for-your-corrected-time-series-xt.-estimate-the-model-parameters-and-check-its-validity}{%
\subsection{4 - Pick (and justify your choice) an ARMA(p,q) model for
your corrected time series Xt. Estimate the model parameters and check
its
validity}\label{pick-and-justify-your-choice-an-armapq-model-for-your-corrected-time-series-xt.-estimate-the-model-parameters-and-check-its-validity}}

\hypertarget{write-the-arimapdq-model-for-the-chosen-series.}{%
\subsection{5 - Write the ARIMA(p,d,q) model for the chosen
series.}\label{write-the-arimapdq-model-for-the-chosen-series.}}

\hypertarget{part-iii-prediction}{%
\section{Part III : Prediction}\label{part-iii-prediction}}

\hypertarget{write-the-equation-verified-by-the-confidence-region-of-level-alpha-on-the-future-values-xt1xt2.}{%
\subsection{\texorpdfstring{6 - Write the equation verified by the
confidence region of level \(\alpha\) on the future values
(XT+1,XT+2).}{6 - Write the equation verified by the confidence region of level \textbackslash alpha on the future values (XT+1,XT+2).}}\label{write-the-equation-verified-by-the-confidence-region-of-level-alpha-on-the-future-values-xt1xt2.}}

\hypertarget{give-the-hypotheses-used-to-get-this-region.}{%
\subsection{7 - Give the hypotheses used to get this
region.}\label{give-the-hypotheses-used-to-get-this-region.}}

\hypertarget{graphically-represent-this-region-for-alpha-95.-comment-on-it.}{%
\subsection{\texorpdfstring{8 - Graphically represent this region for
\(\alpha\) = 95\%. Comment on
it.}{8 - Graphically represent this region for \textbackslash alpha = 95\%. Comment on it.}}\label{graphically-represent-this-region-for-alpha-95.-comment-on-it.}}

\hypertarget{open-question-let-yt-a-stationary-time-series-available-from-t-1-to-t.-we-assume-that-yt1-is-available-faster-than-xt1.-under-which-conditions-does-this-information-allow-you-to-improve-the-prediction-of-xt1-how-would-you-test-itthem}{%
\subsection{9 - Open question : let Yt a stationary time series
available from t = 1 to T. We assume that YT+1 is available faster than
XT+1. Under which condition(s) does this information allow you to
improve the prediction of XT+1 ? How would you test it/them
?}\label{open-question-let-yt-a-stationary-time-series-available-from-t-1-to-t.-we-assume-that-yt1-is-available-faster-than-xt1.-under-which-conditions-does-this-information-allow-you-to-improve-the-prediction-of-xt1-how-would-you-test-itthem}}

\end{document}
